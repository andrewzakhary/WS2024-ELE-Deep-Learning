\documentclass[conference]{IEEEtran}
\IEEEoverridecommandlockouts
% The preceding line is only needed to identify funding in the first footnote. If that is unneeded, please comment it out.
\usepackage{cite}
\usepackage{amsmath,amssymb,amsfonts}
\usepackage{algorithmic}
\usepackage{graphicx}
\usepackage{textcomp}
\usepackage[table]{xcolor}
\usepackage{xcolor}
\def\BibTeX{{\rm B\kern-.05em{\sc i\kern-.025em b}\kern-.08em
    T\kern-.1667em\lower.7ex\hbox{E}\kern-.125emX}}
\begin{document}

\title{Deep Reinforcement Learning usage in Games Playing\
}

\author{\IEEEauthorblockN{ Andrew Zakhary}
\IEEEauthorblockA{\textit{Electronics Engineering} \\
\textit{Hochschule Hamm-Lippstadt}\\
Hamm, Germany \\
2210009}
}





\maketitle

\begin{abstract}

\end{abstract}

\begin{IEEEkeywords}
Deep learning, Reinforcement learning, artificial intelligence, video games
\end{IEEEkeywords}

\section{Introduction}
Games have been a large part of our human society and history from the beginning. Evidence was found that early humans used $Talus bones$ as rudimentary dice as early as 10,000 years B.C\cite{10.1007/978-981-10-0575-6_1     }. As human society developed so did the games they played. The first board game found was in the Levant, dating back to around 7,500 years ago \cite{simpson2007earliest} with a couple of rows of holes carved into limestone. from then on board games developed more complexity from the relative simplicity of games as $Mancala$ in the Mediterranean to $Senet$ in Egypt to $Go$ in chine, which is the  oldest board game of mental skill in the world that is still being played\cite{shotwell1994game}. Games like $Go$ and $Chess$ (previously $chatarunga$) captivated the minds of humans for centuries because they are so complex that no one human could ever master them. This is due to the vast number of possible variations in a single game. In chess for example a typical game is around 40 moves, this could lead $10^{120}$ possible positions according to Claude Shanon\cite{shannon1950xxii}.

In the last decades though, games have found their renaissance with the new technological revolution. The invention of the computer allowed games to take on a new shape. It's highly argued what could count as the first $video game$ because the exact definition of the word $game$ is contested. what could be said with certainty though is that video games started around the 1950s in research facilities. After that video games became commercially available such as $Computer Space$ in 1971 and $Pong$ in 1971
\section{Basic Concepts}
\subsection{Reinforcement learning}
Agent perfroms Actions in the Environment, gets Reward and calculates the next Action
signal (S,a,r) are called experience. 
at max time the set of experiences collected is called trajectory, the period of time is called an episode.

\section{Implementation}

\subsection{Methodology}

\subsection{Example}

\section{Conclusion}
\bibliographystyle{unsrt} 
\bibliography{refs}

\end{document}
